\documentclass{lsanche_cv}

% ATS compat
\input{glyphtounicode}
\pdfgentounicode=1


\usepackage[letterpaper, total={7in, 9.5in}]{geometry}
\usepackage{multicol}
\usepackage{enumitem}
\usepackage[none]{hyphenat}

% Some font info
\usepackage[T1]{fontenc}
\usepackage[utf8]{inputenc}
\usepackage[default]{lato}

\setlist[itemize]{leftmargin=*}

\begin{document}
\pagenumbering{gobble}
\name{Lyndon Sanche}

\medskip

\tagline{BSc. Electrical Engineering -- EIT}

\smallskip

\contact{780-940-3401}{lsanche@lyndeno.ca}{lyndeno.ca}{linkedin.com/in/LyndonSanche}{github.com/Lyndeno}

\section{Summary}
Highly motivated and skilled Electrical Engineer with experience in software development, system administration, and infrastructure protection. Proven ability to develop and implement solutions using various programming languages and tools, with a strong commitment to open-source contributions and continuous learning.

\section{Education}
\eduexp{University of Alberta}{September 2017 -- April 2021}{Bachelor of Science in Electrical Engineering}{Edmonton, Alberta, Canada}

\section{Employment}
\workexp{Data Engineer}{DataAnnotation}{DataAnnotation}{January 2025 -- Present}{Remote}
\begin{itemize}
\item Evaluated over several LLM responses daily, ensuring accuracy and alignment with instructions, contributing to continuous improvement in model reliability.
\item Developed and implemented over 50 complex code snippets in Rust, C, Javascript, SQL, and Python to guide AI response generation, resulting in a continuous increase in code accuracy.
\item Identified and flagged potentially unsafe or malicious AI responses, contributing to a safer and more reliable LLM.
\end{itemize}

\divider

\workexp{Engineer in Training -- CIP Compliance}{Protection and Control \& Operations Engineering}{ATCO Electric}{July 2022 -- Present}{Edmonton, Alberta, Canada}
\begin{itemize}
\item Developed and deployed a PowerApp to recognize positive safety practices, resulting in an increase in employee engagement with safety protocols.
\item Automated the generation of relay setting deficiency reports using SQL, PowerBI, and Power Automate, reducing report creation time from 2 hours to 5 minutes per week.
\item Managed and ensured compliance with Critical Infrastructure Protection (CIP) standards across multiple projects.
\end{itemize}

\divider

\workexp{Engineer in Training}{Protection and Control}{ATCO Electric}{November 2021 -- July 2022}{Edmonton, Alberta, Canada}
\begin{itemize}
\item Developed and delivered a PowerPoint presentation on Driving Safety to 100 team members, resulting in increased awareness of driving dangers.
\item Investigated faulty battery monitoring systems and collaborated with vendor to improve substation backup power reliability.
\end{itemize}

\divider

\workexp{Engineer in Training}{Research \& Development}{Eleven Engineering}{June -- November 2021}{Edmonton, Alberta, Canada}
\begin{itemize}
\item Developed firmware in C++ with Assembly for the Xinc2 platform, contributing to wireless audio reliability and physical user interface functionality.
\item Debugged a wireless signal timing error in Assembly code, resolving a problem with audio cutting out due to an extra line of assembly causing the signal timing to be off.
\end{itemize}

\section{Other Experience}
\workexp{Kernel Maintainer}{Dell-PC Module}{Linux Kernel}{May 2024 -- Present}{Remote}

\noindent{I improved Dell PC support in the Linux kernel by creating a link between the Dell BIOS fan modes and the kernel platform\_profile interface. This work enhanced user experience by linking with already existing user tools to make it easy to change the fan mode on the computer. Written in C.}

\divider

\workexp{Package Maintainer}{NixOS}{nixpkgs}{July 2024 -- Present}{Remote}

\noindent{Maintain and review updates to packages in the NixOS Linux distribution. Packages maintained include lubelogger, a vehicle maintenance tracker; and power-profiles-daemon, a Linux power management service.}

\divider

\workexp{Open Source Developer}{Various Projects}{}{Ongoing}{Remote}

\noindent{I actively use my free time to develop open source software to solve problems and hone my skills. I am currently working on ironfetch, a system information tool; ppd-rs, a Rust library for the power-profiles-daemon; memdev, a Rust library for getting system memory information; and more recently, lowpowerd, a small utility to automatically set low power modes on Linux. My latest interest has been in build automation, utilizing Github actions and Nix Hydra for continuous integration.}

\section{Projects}

	\project{Dell-PC Kernel Module}
		\begin{itemize}
			\item I created a new module upstreamed to the Linux kernel in version 6.11.
			\item I enhanced Linux usability by linking the kernel platform\_profile interface to Dell computer fan modes controlled in the smbios tables.
      \item I enhanced user experience by allowing these fan modes to be configured through an easy to use interface.
		\end{itemize}

	\divider

\project{Mesh Messaging System}
\begin{itemize}
\item Electrical engineering capstone project utilizing Microchip Atmel and Espressif chipsets.
\item Mesh communication system to communicate over a kilometer or more. 
\item I designed an app that allowed phones to connect to the mesh points over wifi. I implemented the system so that messages sent to users travelled through the mesh points.
\item This project was a proof of concept for a system that can allow offgrid communication over a large distance.
\item I learned to program C with the FreeRTOS project, gaining knowledge of RTOS programming in embedded environments.
\end{itemize}

\divider

\project{Personal NixOS Machine}
\begin{itemize}
\item NixOS for both the personal computer and server functions. Configured in the Nix language.
\item I implemented the usage of ZFS for data redundancy, with offsite backups for safety.
\item I enhanced security and usability for my other machines by hosting a Hydra server to automatically build my machine configurations before pushing them to a Minio S3 cache using attic. This allows my other, slower, machines to be upgraded effortlessly.
\item I improved privacy by running ollama locally, sharing access with my phone and laptop for easy large language model usage without privacy concerns.
\item Nix Hydra server is running to continuously build and test all of my projects.
\end{itemize}

\section{Technical Skills}
\begin{itemize}
\item MATLAB, Bentley Microstation, Xilinx Vivado, MS Office, Microchip AVR
\item C, Rust, Nix/NixOS, Powershell, VBA, FreeRTOS, AVR C, HTML, Python
\item Linux Kernel development, from writing code to sending to mailing list.
\item NERC CIP (Critical Infrastructure Protection)
\end{itemize}

\end{document}
