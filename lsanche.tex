\documentclass{lsanche_cv}

% ATS compat
\input{glyphtounicode}
\pdfgentounicode=1


\usepackage[letterpaper, total={7in, 9.5in}]{geometry}
\usepackage{multicol}
\usepackage{enumitem}
\usepackage[none]{hyphenat}

% Some font info
\usepackage[T1]{fontenc}
\usepackage[utf8]{inputenc}
\usepackage[default]{lato}

\setlist[itemize]{leftmargin=*}

\begin{document}
\pagenumbering{gobble}
\name{Lyndon Sanche}

\medskip

\tagline{BSc. Electrical Engineering -- Engineer in Training (E.I.T.)}

\smallskip

\contact{780-940-3401}{lsanche@lyndeno.ca}{lyndeno.ca}{linkedin.com/in/LyndonSanche}{github.com/Lyndeno}

\section{Education}
\eduexp{University of Alberta}{September 2017 -- April 2021}{BSc. Electrical Engineering}{Edmonton, Alberta}

\section{Employment}
\workexp{Data Engineer}{DataAnnotation}{DataAnnotation}{January 2025 -- Present}{Remote}
\begin{itemize}
		\item Helped train large language models by rating responses by correctness and instruction following.
		\item Enhanced AI coding ability by writing complex code in Rust, C, Javascript, SQL, and Python to guide AI responses.
		\item Enhanced AI understanding of intent by creating fine-grained criteria to rate AI responses againt.
		\item Encouraged safety in AI by marking responses that are incorrect or malicious.
\end{itemize}

\divider

\workexp{Engineer in Training -- CIP Compliance}{Protection and Control \& Operations Engineering}{ATCO Electric}{July 2022 -- Present}{Edmonton, Alberta}
\begin{itemize}
		\item Ensure CIP (Critical Infrastructure Protection) compliance by:
    \begin{itemize}
      \item Completing management-of-change procedures.
      \item Maintaining baseline records.
      \item Updating Electronic Security Perimeter drawings.
      \item Completing monthly patch reports.
      \item Creating and completing mitigation plan tasks.
      \item Act as CIP resource for affected projects.
    \end{itemize}
		\item Prepare monthly As-Build and Issue reports.
		\item Manage PowerApp for recognising positive safety practice.
		\item Create automated relay setting deficiency report using SQL, PowerBI, and Power Automate.
\end{itemize}

\divider

\workexp{Engineer in Training}{Protection and Control}{ATCO Electric}{November 2021 -- July 2022}{Edmonton, Alberta}
\begin{itemize}
		\item Assist with regulation compliance by updating entries in tracking spreadsheets and Maximo database.
		\item Resolve outstanding as-builts in relay settings database.
		\item Prepared and presented PowerPoint to team members on safety topic.
		\item Collect and analyze relay records when outages occur.
\end{itemize}

\divider

\workexp{Engineer in Training}{Research \& Development}{Eleven Engineering}{June -- November 2021}{Edmonton, Alberta}
\begin{itemize}
	\item Mobile and desktop app to interface with embedded wireless audio hardware. Utilizing Qt with experimental Rust rewrite.
	\item Firmware development in C++ with a touch of Assembly for the Xinc2 platform.
	\item Developed rotary encoder driver for volume control.
	\item Debugged wireless signal to fix timing error in Assembly code.
\end{itemize}

\section{Other Experience}
\workexp{Kernel Maintainer}{Dell-PC Module}{Linux Kernel}{May 2024 -- Present}{Remote}

\noindent{I improved Dell PC support in the Linux kernel by creating a link between the Dell BIOS fan modes and the kernel platform\_profile interface. This work enhanced user experience by linking with already existing user tools to make it easy to change the fan mode on the computer. Written in C.}

\divider

\workexp{Package Maintainer}{NixOS}{nixpkgs}{July 2024 -- Present}{Remote}

\noindent{Maintain and review updates to packages in the NixOS Linux distribution. Packages maintained include lubelogger, a vehicle maintenance tracker; and power-profiles-daemon, a Linux power management service.}

\divider

\workexp{Open Source Developer}{Various Projects}{}{Ongoing}{Remote}

\noindent{I actively use my free time to develop open source software to solve problems and hone my skills. I am currently working on ironfetch, a system information tool; ppd-rs, a Rust library for the power-profiles-daemon; memdev, a Rust library for getting system memory information; and more recently, lowpowerd, a small utility to automatically set low power modes on Linux. My latest interest has been in build automation, utilizing Github actions and Nix Hydra for continuous integration.}

\section{Projects}

	\project{Dell-PC Kernel Module}
		\begin{itemize}
			\item I created a new module upstreamed to the Linux kernel in version 6.11.
			\item I enhanced Linux usability by linking the kernel platform\_profile interface to Dell computer fan modes controlled in the smbios tables.
      \item I enhanced user experience by allowing these fan modes to be configured through an easy to use interface.
		\end{itemize}

	\divider

\project{Mesh Messaging System}
\begin{itemize}
\item Electrical engineering capstone project utilizing Microchip Atmel and Espressif chipsets.
\item Mesh communication system to communicate over a kilometer or more. 
\item I designed an app that allowed phones to connect to the mesh points over wifi. I implemented the system so that messages sent to users travelled through the mesh points.
\item This project was a proof of concept for a system that can allow offgrid communication over a large distance.
\item I learned to program C with the FreeRTOS project, gaining knowledge of RTOS programming in embedded environments.
\end{itemize}

\divider

\project{Personal NixOS Machine}
\begin{itemize}
\item NixOS for both the personal computer and server functions. Configured in the Nix language.
\item I implemented the usage of ZFS for data redundancy, with offsite backups for safety.
\item I enhanced security and usability for my other machines by hosting a Hydra server to automatically build my machine configurations before pushing them to a Minio S3 cache using attic. This allows my other, slower, machines to be upgraded effortlessly.
\item I improved privacy by running ollama locally, sharing access with my phone and laptop for easy large language model usage without privacy concerns.
\item Nix Hydra server is running to continuously build and test all of my projects.
\end{itemize}

\section{Technical Skills}
\begin{itemize}
\item MATLAB, Bentley Microstation, Xilinx Vivado, MS Office, Microchip AVR
\item C, Rust, Nix/NixOS, Powershell, VBA, FreeRTOS, AVR C, HTML
\item Linux Kernel development, from writing code to sending to mailing list.
\item NERC CIP (Critical Infrastructure Protection)
\end{itemize}

\end{document}
